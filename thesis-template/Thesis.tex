%% ----------------------------------------------------------------
%% Thesis.tex -- MAIN FILE (the one that you compile with LaTeX)
%% ---------------------------------------------------------------- 

% Set up the document
\documentclass[a4paper, 11pt, twoside, openright]{Thesis}  % Use the "Thesis" style, based on the ECS Thesis style by Steve Gunn
\graphicspath{{res/Figures/}}  % Location of the graphics files (set up for graphics to be in PDF format)

% Include any extra LaTeX packages required
\usepackage[square, numbers, comma, sort&compress]{natbib}  % Use the "Natbib" style for the references in the Bibliography
\usepackage{verbatim}  % Needed for the "comment" environment to make LaTeX comments
\usepackage{vector}  % Allows "\bvec{}" and "\buvec{}" for "blackboard" style bold vectors in maths
\usepackage{algorithm}
\usepackage{algorithmic}
\usepackage[T1]{fontenc}

\usepackage{subcaption}
\usepackage{venturis2}
\usepackage{lmodern}
\usepackage{textcomp}    % solve issues with lmodern
\usepackage{amsfonts}
\usepackage{amsmath}
\usepackage{amsthm}
\usepackage{mathrsfs}
\usepackage{amssymb}
\usepackage{wasysym}
\usepackage{microtype}   % better typesetting with pdfLaTeX
\usepackage{ps4pdf}

\usepackage{verbatim}
\usepackage{algorithm}
\usepackage{algorithmic}
\usepackage{graphicx}
\usepackage{caption}
\usepackage{mathtools}
\usepackage{color}

\usepackage{booktabs}
\usepackage{multirow}
\usepackage{blindtext}

\usepackage{csvsimple}
\usepackage[flushleft]{threeparttable}

\usepackage[english]{babel}
\addto\extrasenglish{
  \def\subsectionautorefname{Subsection}
  \def\sectionautorefname{Section}
}

\newcommand*\justify{%
  \fontdimen2\font=0.4em% interword space
  \fontdimen3\font=0.2em% interword stretch
  \fontdimen4\font=0.1em% interword shrink
  \fontdimen7\font=0.1em% extra space
  \hyphenchar\font=`\-% allowing hyphenation
}
\title{\LaTeX}

\date{}

\newtheorem{mydef}{Definition}
\newtheorem{myex}{Example}
\newtheorem{myprob}{Problem}
\PSforPDF{
  \usepackage{pstricks}
}
\usepackage{caption}
\usepackage[pdftex,dvipsnames]{xcolor}
\usepackage[compact]{titlesec}
\usepackage{booktabs}
\usepackage{sectsty}     % section titles in specified font face
%\usepackage{listings}
\usepackage{listingsutf8}
\lstset{
  language=Java,
  keywordstyle=\color{blue},
  commentstyle=\color{gray},
  stringstyle=\color{orange},
  basicstyle=\ttfamily\small,
  numbers=left,
  numberstyle=\tiny,
  breaklines=true,
}

\definecolor{delim}{RGB}{20,105,176}
\definecolor{numb}{RGB}{106, 109, 32}
\definecolor{string}{rgb}{0.64,0.08,0.08}

\lstdefinelanguage{json}{
    numbers=left,
    numberstyle=\small,
    frame=single,
    rulecolor=\color{black},
    showspaces=false,
    showtabs=false,
    breaklines=true,
    postbreak=\raisebox{0ex}[0ex][0ex]{\ensuremath{\color{gray}\hookrightarrow\space}},
    breakatwhitespace=true,
    basicstyle=\ttfamily\small,
    upquote=true,
    morestring=[b]",
    stringstyle=\color{string},
    literate=
     *{0}{{{\color{numb}0}}}{1}
      {1}{{{\color{numb}1}}}{1}
      {2}{{{\color{numb}2}}}{1}
      {3}{{{\color{numb}3}}}{1}
      {4}{{{\color{numb}4}}}{1}
      {5}{{{\color{numb}5}}}{1}
      {6}{{{\color{numb}6}}}{1}
      {7}{{{\color{numb}7}}}{1}
      {8}{{{\color{numb}8}}}{1}
      {9}{{{\color{numb}9}}}{1}
      {\{}{{{\color{delim}{\{}}}}{1}
      {\}}{{{\color{delim}{\}}}}}{1}
      {[}{{{\color{delim}{[}}}}{1}
      {]}{{{\color{delim}{]}}}}{1},
}

%for todos 
  % for disabling notes when printing change to \usepackage[disable]{todonotes}
\usepackage{xargs}
\usepackage[colorinlistoftodos,prependcaption,textsize=tiny, disable]{todonotes}
\setlength{\marginparwidth}{2.1cm}  % Match this to the width you want
\newcommandx{\unsure}[2][1=]{\todo[linecolor=red,backgroundcolor=red!25,bordercolor=red,#1]{#2}}
\newcommandx{\change}[2][1=]{\todo[linecolor=blue,backgroundcolor=blue!25,bordercolor=blue,#1]{#2}}
\newcommandx{\info}[2][1=]{\todo[linecolor=OliveGreen,backgroundcolor=OliveGreen!25,bordercolor=OliveGreen,#1]{#2}}
\newcommandx{\improvement}[2][1=]{\todo[linecolor=Plum,backgroundcolor=Plum!25,bordercolor=Plum,#1]{#2}}
\newcommandx{\thiswillnotshow}[2][1=]{\todo[disable,#1]{#2}}
%end for todos

\usepackage{enumitem}
\newcommand\litem[1]{\item{\bfseries #1}} % for labeled items (see tools used)

\allsectionsfont{\sffamily}
\numberwithin{algorithm}{chapter}
\setcounter{secnumdepth}{3}
\setcounter{tocdepth}{2}
\renewcommand{\captionlabelfont}{\sffamily\bfseries}
\newtheorem{thm}{Theorem}
\renewcommand{\algorithmicrequire}{\textbf{Input:}}
\renewcommand{\algorithmicensure}{\textbf{Output:}}
\newcommand{\term}[1]{{\sf {\small #1}}}
\hypersetup{urlcolor=black, colorlinks=false}  % Colours hyperlinks in blue, but this can be distracting if there are many links.
\input{src/definitions}


\listfiles

%\usepackage[draft]{hyperref}
%\usepackage[hyperfootnotes=false,plainpages=false]{hyperref}
%% ----------------------------------------------------------------
\begin{document}
\frontmatter	  % Begin Roman style (i, ii, iii, iv...) page numbering

% Set up the Title Page
\title  {Byetrack: Capabilities as a Solution against Tracking Across Android Apps} \thiswillnotshow{give it a title}
\authors  {Tim Christmann}
\addresses  {\groupname\\\deptname\\\univname}  % Do not change this here, instead these must be set in the "Thesis.cls" file, please look through it instead
\UNIVERSITY {\texorpdfstring{\href{https://www.uni-saarland.de}
            {Universit\"at des Saarlandes}}
            {Universit\"at des Saarlandes}}
\faculty{MI Fakult\"at f\"ur Mathematik und Informatik}
\department{\texorpdfstring{\href{https://saarland-informatics-campus.de/}
            {Department of Computer Science}}
            {Department of Computer Science}}
\reviewers{Dr. Sven Bugiel}{Prof. Dr. Andreas Zeller}
\date       {\today}
\subject    {}
\keywords   {}
\thesistype {Bachelorthesis}
\handin     {November 20, 2025}
\ifdefined\texhash
    \version {\texhash}
\else
    \version{}
\fi
\maketitle
%% ----------------------------------------------------------------

\setstretch{1.3}  % It is better to have smaller font and larger line spacing than the other way round

% Define the page headers using the FancyHdr package and set up for one-sided printing
%\fancyhead{}  % Clears all page headers and footers
%\rhead{\thepage}  % Sets the right side header to show the page number
%\lhead{}  % Clears the left side page header

\pagestyle{fancy}  % Finally, use the "fancy" page style to implement the FancyHdr headers
\fancyhead[RE,LO]{\sffamily\bfseries\nouppercase{\rightmark}}
\fancyhead[LE,RO]{\thepage}
%% ----------------------------------------------------------------
% Declaration Page required for the Thesis, your institution may give you a different text to place here
%\vspace*{1cm}
%\textbf{\large Eidesstattliche Erkl\"arung}\\[1em]
%Ich erkl\"are hiermit an Eides statt, dass ich die vorliegende Arbeit selbst\"andig verfasst 
%und keine anderen als die angegebenen Quellen und Hilfsmittel verwendet habe. 
%\\[0.3cm]
%
%\textbf{\large Statement in Lieu of an Oath}\\[1em]
%I hereby confirm that I have written this thesis on my own and that I have not used any other media or materials than the ones referred to in this thesis.
%
%Saarbr\"ucken, \handindate,\\[1.5cm]
%\hspace*{1cm}(\authornames)\\[2cm]
%
%\textbf{\large Einverst\"andniserkl\"arung}\\[1em]
%Ich bin damit einverstanden, dass meine (bestandene) Arbeit in beiden Versionen in die Bibliothek der Informatik aufgenommen und damit ver\"offentlicht wird.
%\\[0.3cm]
%
%\textbf{\large Declaration of Consent}\\[1em]
%I agree to make both versions of my thesis (with a passing grade) accessible to the public by having them added to the library of the Computer Science Department.
%
%Saarbr\"ucken, \handindate,\\[1.5cm]
%\hspace*{1cm}(\authornames)

\section*{Eidesstattliche Erklärung}
% Set this macro to the correct value for YOUR thesis
\newcommand{\teilaufgaben}{{\textbf{code generation, literature research, text rewriting/revision}}}
Hiermit erkläre ich, dass ich die vorliegende Arbeit selbstständig und ohne die Beteiligung dritter Personen verfasst habe, und dass ich keine anderen als die angegebenen Quellen und Hilfsmittel benutzt habe.
 Alle Stellen der Arbeit, die wörtlich oder sinngemäß aus Veröffentlichungen oder aus anderweitigen fremden Äußerungen entnommen wurden, sind als solche kenntlich gemacht.
 Insbesondere bestätige ich hiermit, dass ich bei der Erstellung der nachfolgenden Arbeit mittels künstlicher Intelligenz betriebene Software (z. B. ChatGPT) ausschließlich für folgende zulässige Teilaufgaben: \teilaufgaben, und nicht zur Bearbeitung der in der Arbeit aufgeworfenen Fragestellungen zu Hilfe genommen habe.
 Alle mittels künstlicher Intelligenz betriebenen Software (z. B. ChatGPT) generierten und/oder bearbeiteten Teile der Arbeit wurden kenntlich gemacht und als Hilfsmittel angegeben.
 Mir ist bewusst, dass der Verstoß gegen diese Versicherung zum Nichtbestehen der Prüfung bis hin zum Verlust des Prüfungsanspruchs führen kann.




\section*{Declaration of Original Authorship}
% Set this macro to the correct value for YOUR thesis
\newcommand{\tasks}{{\textbf{code generation, literature research, text rewriting/revision}}}
I hereby declare that this thesis is my own original work and was completed independently, without unauthorized assistance or unacknowledged sources.
Any content derived from publications or other external sources, whether quoted verbatim or paraphrased, has been duly acknowledged and clearly marked as such.
I hereby confirm that, in preparing the following thesis, I have used artificial intelligence-based software (such as ChatGPT) solely for the following permitted tasks: \tasks, and not to address the core research questions presented in this thesis.
All parts of the thesis produced or modified with the assistance of artificial intelligence-based software (such as ChatGPT) have been clearly indicated as such and that the software used has been listed as a resource.
I acknowledge that any breach of this declaration may result in failing the examination and, in severe cases, the right to be examined may be revoked.





\vspace*{\fill}

\noindent\begin{tabular}{ll}
\makebox[2.5in]{\hrulefill} & \makebox[2.5in]{\hrulefill}\\
Ort/Place, Datum/Date & Unterschrift/Signature\\[8ex]
\end{tabular}
\cleardoublepage

% END OF frontmatter.tex

%% ----------------------------------------------------------------
% The "Dedication Page"
%\pagestyle{empty}  % No headers or footers for the following pages

%\null\vfill\vfill\vfill\vfill\vfill
% Now comes the "Dedication Page", written in italics
\todo[inline]{What is the dedication page?}
Yes, you can

\vfill\vfill\vfill\vfill\vfill\vfill\null
\cleardoublepage  % Dedication page ended, start a new page


%% ----------------------------------------------------------------
\pagestyle{empty}

\mbox{}
%\clearpage
\setstretch{1.3}  % Reset the line-spacing to 1.3 for body text (if it has changed)

% The Abstract Page
\addtotoc{Abstract}  % Add the "Abstract" page entry to the Contents
\abstract{

\todo[inline]{Write the abstract.}
% -----------------------------------------------------------



% -----------------------------------------------------------
\addtocontents{toc}{\vspace{1em}}  % Add a gap in the Contents, for aesthetics

\clearpage  % Abstract ended, start a new page

%% ----------------------------------------------------------------
\pagestyle{empty}
\mbox{}
\clearpage
\setstretch{1.3}  % Reset the line-spacing to 1.3 for body text (if it has changed)

% The Acknowledgments page, for thanking everyone
\acknowledgements{
\addtocontents{toc}{\vspace{1em}}  % Add a gap in the Contents, for aesthetics

% ------------------------- Acknowledgments -------------------------

I would like to express my sincere gratitude to Noah Mauthe for supervising this thesis and for his continuous guidance, support, and technical insight.
His willingness to help, his clear and constructive feedback, and the many discussions we had were essential for shaping both the ideas and the implementation presented in this work.
I am deeply thankful for his commitment throughout the entire project.

I would also like to thank Dr. Sven Bugiel for giving me the opportunity to conduct this thesis in his group and for his valuable input and encouragement during the project.

% -------------------------------------------------------------------

}
\clearpage
% End of the Acknowledgments

%% ----------------------------------------------------------------

\pagestyle{fancy}  %The page style headers have been "empty" all this time, now use the "fancy" headers as defined before to bring them back



%% ----------------------------------------------------------------
%\lhead{\emph{Contents}}  % Set the left side page header to "Contents"
\tableofcontents  % Write out the Table of Contents

%\lhead{\emph{List of Figures}}  % Set the left side page header to "List if Figures"
\listoffigures  % Write out the List of Figures

%% ----------------------------------------------------------------
%\lhead{\emph{List of Tables}}  % Set the left side page header to "List of Tables"
\listoftables  % Write out the List of Tables


%% ----------------------------------------------------------------
\setstretch{1.5}  % Set the line spacing to 1.5, this makes the following tables easier to read
\clearpage  % Start a new page
%\lhead{\emph{Abbreviations}}  % Set the left side page header to "Abbreviations"
%\listofsymbols{ll}  % Include a list of Abbreviations (a table of two columns)
%{
% \textbf{Acronym} & \textbf{W}hat (it) \textbf{S}tands \textbf{F}or \\
%\textbf{LAH} & \textbf{L}ist \textbf{A}bbreviations \textbf{H}ere \\

%}

%% ----------------------------------------------------------------
%\clearpage  %Start a new page
%\lhead{\emph{Symbols}}  % Set the left side page header to "Symbols"
%\listofnomenclature{lll}  % Include a list of Symbols (a three column table)
%{
% symbol & name & unit \\
%$a$ & distance & m \\
%$P$ & power & W (Js$^{-1}$) \\
%& & \\ % Gap to separate the Roman symbols from the Greek
%$\omega$ & angular frequency & rads$^{-1}$ \\
%}
%% ----------------------------------------------------------------
% End of the pre-able, contents and lists of things
% Begin the Dedication page

\setstretch{1.3}  % Return the line spacing back to 1.3

%\pagestyle{empty}  % Page style needs to be empty for this page

\addtocontents{toc}{\vspace{2em}}  % Add a gap in the Contents, for aesthetics


%% ----------------------------------------------------------------
\mainmatter	  % Begin normal, numeric (1,2,3...) page numbering
\pagestyle{fancy}  % Return the page headers back to the "fancy" style

% Include the chapters of the thesis, as separate files
% Just uncomment the lines as you write the chapters




%%%%%%%%%%%%%%%%%%%%%%%%%%%%%%%%%%%%%%%%%%%%%%%%%%%%%%%%%%%%%%%%%%%%%%%%%%%%%%%%%%%%%%%%%%%%%%%%%%%%%%%%%%%%%%%%%%%%%%%%%%%%%%%%%

\chapter{Introduction}

\todo[inline]{Write the introduction.}

% An introduction into the field, the problem (including the research questions), and the proposed solution.


\blindtext[1]

% Example figue
		\begin{figure}
			\lstinputlisting[language=C, firstline=\interestingstart, lastline=\interestingend]{\somecfile}
			\caption{caption}
			\label{code:aes_unsealdata}
		\end{figure}



 % Introduction

\chapter{Background}

% Provide the necessary technical and conceptual background:

% - Overview of Custom Tabs (CTs), Trusted Web Activities (TWAs), and browser cookie sharing on Android.

% (- Explain HyTrack in more detail (summarize its attack flow, key findings, and impact) — this links to your implementation context.)

% (- Introduce Capabilities/Tokens?)

Before presenting the design and implementation of our mitigation, we first provide the necessary background on web tracking (\autoref{sec:tracking-web-mobile}), Android's web content integration mechanisms (\autoref{sec:customtabs-twas}), and the HyTrack attack (\autoref{sec:hytrack}).
Then, we introduce the concept of capabilities and capability-based security (\autoref{sec:capabilities}), which forms the basis of our approach (\autoref{sec:our-approach}).
Finally, we summarize the threat model we assume throughout this thesis (\autoref{sec:threat_model}).


\section{Tracking on the Web and Mobile}
\label{sec:tracking-web-mobile}

Tracking mechanisms are typically divided into two broad categories: stateful and stateless tracking.
Stateless tracking, also known as fingerprinting, infers a user's identity based on a combination of device-specific attributes ~\cite{laperdrix2020browser}.
Consequently, this method is hard to detect and block, but is also inherently less reliable, as small system changes may alter the fingerprint and disrupt identification.

Instead, stateful tracking relies on storing unique identifiers on the client device, most commonly through cookies or local storage.
When a user revisits a site or interacts with embedded third-party content across domains, these identifiers are sent along with requests, allowing persistent recognition.
While straightforward and highly effective, stateful tracking has become increasingly restricted through browser policies (e.g., third-party cookie blocking) and mobile platform changes such as the ability to disable the Google Advertising ID (GAID) on Android ~\cite{google:gaid}.

% --------------------Other tracking attacks (Facebook localhost leak)--------------------------
This problem not only affects the web, but also extends into the mobile ecosystem, as recently demonstrated by the Facebook Localhost Scandal~\cite{localleaks} that exposed a covert tracking method used by Meta and Yandex on Android.
In this case, their apps (e.g., Instagram) silently listened on localhost ports to receive browser tracking data, such as mobile browsing sessions and web cookies, sent from websites embedding Meta Pixel or Yandex scripts.
This allowed the apps to link web activity to logged-in users, bypassing the browser's and Android's privacy protections.
Although the practice was discontinued shortly after public disclosure, it highlighted a critical privacy gap between web content and native apps on mobile platforms.

\section{Custom Tabs and Trusted Web Activities on Android}
\label{sec:customtabs-twas}

\begin{table}[h]
\centering
\begin{threeparttable}

\begin{tabular}{@{}lcccc@{}}

\toprule
Capability & Browser & WebView & Custom Tab & Trusted Web Activity \\ 
\midrule
Integration possible  & \Circle & \CIRCLE & \CIRCLE & \CIRCLE \\
Shares browser state  & \CIRCLE \tnote{1} & \Circle & \CIRCLE & \CIRCLE \\
Can hide URL bar      & \Circle & \Circle & \Circle & \CIRCLE \\
Can control size      & \Circle & \CIRCLE & \LEFTcircle \tnote{2} & \Circle \\
Can open any URL      & \CIRCLE & \CIRCLE & \CIRCLE & \Circle \tnote{3} \\
\bottomrule
\end{tabular}
\Circle: unsupported, \CIRCLE: supported, \LEFTcircle: partially supported, 1: Owns the shared state, 2: Custom Tabs can be reduced to 50\% programmatically and minimized by users, 3: Requires associatation via asset links.
\end{threeparttable}
\caption{Overview of Web Content Integration Mechanisms on Android. \\(Taken from Wessel et al.'s HyTrack paper~\cite{USENIX:Wessels:2025})}
\label{tab:webcontent-overview}
\end{table}

To integrate web content into Android applications, developers can use several mechanisms that differ in terms of security, performance, and user experience (\autoref{tab:webcontent-overview}).
Among these, Custom Tabs (CTs) ~\cite{android:customtabs} and Trusted Web Activities (TWAs) ~\cite{android:trustedwebactivities} have emerged as the most popular alternatives to traditional WebViews ~\cite{android:webviews}, offering better performance and tighter integration with the user's default browser.

CTs were introduced to allow apps to display web content within the app's interface while leveraging the full capabilities of the user's browser.
Unlike a WebView, which runs a separate, minimal web engine within the app, a CT is rendered by the installed browser itself.
This means that all browser features such as optimized rendering, password managers, saved credentials, and cookies remain available.
Developers can also customize the browser's UI elements, such as toolbar color and menu items, to visually align the CT with their app's theme.
As a result, users perceive a seamless transition between native and web content without leaving the app context.

TWAs extend this concept further by removing nearly all browser UI elements, including the URL bar, and displaying web content in full-screen mode.
This allows developers to integrate entire Progressive Web Apps (PWAs) or other web-based experiences into their native apps while maintaining a consistent appearance.
For security reasons, launching a TWA requires a Digital Asset Link (DAL) \cite{android:digitalassetlinks} --- a mutual verification between the app and the website ---, ensuring that both belong to the same trusted party.
If this trust relationship cannot be verified, Android automatically downgrades the TWA to a regular CT.

A key advantage of both CTs and TWAs is that they share the browser's state.
This means users can stay logged in to websites, reuse stored cookies, and maintain personalized settings across different apps and browsing sessions.
This behavior improves usability and supports features like Single Sign-On (SSO), as authentication tokens from the browser can be reused within an app's embedded web view.
However, as later discussed in Section \ref{sec:hytrack}, this same feature also introduces significant privacy risks.
The shared cookie storage allows any app to access browser state information used by others, thereby enabling persistent cross-app and cross-web tracking techniques such as HyTrack.

In summary, CTs and TWAs offer a powerful bridge between the native and web ecosystems on Android.
They combine the convenience and functionality of a full browser with the visual coherence of an app-embedded experience.
Yet, the same integration that improves usability also blurs traditional security and privacy boundaries between apps and the web, which is exploited by for persistent-cross app tracking in the form of the HyTrack attack.

\section{HyTrack Attack Overview}
\label{sec:hytrack}

\begin{figure}[h]
  \centering
  \includegraphics[width=1\textwidth]{hytrack_Flow_colored.pdf}

  \caption{High-level Overview of the HyTrack Attack Flow. \\(Adapted from Wessel et al.'s HyTrack paper~\cite{USENIX:Wessels:2025})}
  \label{fig:hytrack_overview}
\end{figure}

HyTrack, introduced by Wessels et al. \cite{USENIX:Wessels:2025}, exposes a fundamental privacy flaw in this shared-state model.
It demonstrates that third-party libraries embedded in multiple apps can exploit the browser's global cookie storage to identify and track users across applications and even into their normal web browsing sessions.
By leveraging standard Android features, rather than any explicit vulnerability, HyTrack highlights how the very mechanisms designed to make CTs and TWAs seamless for users can also undermine Android's app isolation guarantees.

Whenever an app opens a CT or TWA to display web content, the request is executed within the context of the user’s default browser.
This means that all cookies set by the visited domain are stored in the browser's global cookie jar and automatically reused in subsequent sessions, even if they originate from different apps.
While this shared state enables seamless Single Sign-On and personalization, it also allows a tracking entity to correlate activity across multiple apps that interact with the same web domain.

HyTrack leverages this design as follows (\autoref{fig:hytrack_overview}): a seemingly benign third-party library, included in several independent apps, opens a CT or TWA to a tracking domain controlled by the library's author.
When this web page is first loaded, the server sets a unique identifier in a cookie, which is then stored in the shared browser state.
When another app using the same library later opens a CT or TWA to the same tracking domain, the browser automatically attaches the existing cookie, thereby revealing that both apps are used by the same user.
This creates a powerful cross-app identity link that persists outside Android’s app sandbox and is invisible to both users.

Even more concerning, HyTrack's tracking identifiers are resilient to deletion, because the tracking library can store the identifier in the app's local storage and even use Google Play Services' backup feature to restore it after a device reset.
In effect, HyTrack achieves Evercookie-like persistence at the system level, reviving deleted identifiers upon device restoration.

\subsection{Weaknesses in Custom Tabs and Trusted Web Activities}
The feasibility of this attack stems from three fundamental weaknesses in the current CT and TWA model:

\label{sec:hytrack-weaknesses}
\begin{enumerate}
  \item 
    \textbf{Implicit and Persistent Cookie Sharing:}
    All apps using CTs or TWAs share a single, persistent global browser cookie jar, regardless of developer intent.
    This shared state persists across app launches and user attempts to clear tracking data.

  \item
    \textbf{Lack of App Context in the Browser:}
    The browser has no knowledge of which app initiated a given request and therefore cannot enforce app-specific cookie isolation or policy controls.

  \item
    \textbf{Unrestricted Third-Party Inclusion:}
    Any third-party library that is embedded in multiple apps can launch CTs or TWAs, thereby accessing the shared browser state and enabling cross-app tracking.
\end{enumerate}

By exploiting these design characteristics, HyTrack bridges the isolation between native and web contexts, effectively transforming legitimate web-integration features into a cross-app tracking channel.

\subsection{Goals} 
\label{sec:hytrack-goals}

The authors of HyTrack identified three essential goals that any practical mitigation against cross-app tracking must fulfill~\cite{USENIX:Wessels:2025}.
We adopt these goals as guiding principles for the design of our capability-based framework:

\begin{enumerate}
  \item \textbf{Support for Web Platform Features:}
  Any mitigation should preserve the full functionality of web content, including support for cookies, JavaScript, and modern browser APIs.

  \item \textbf{Seamless Integration:}
  The mitigation must operate transparently, without requiring additional user permissions or altering the normal app and browser experience.

  \item \textbf{Controlled Access to Shared Browser State:}
  Isolation between applications must prevent tracking via shared state, while still allowing legitimate sharing scenarios such as Single Sign-On (SSO).
\end{enumerate}

\subsection{Mitigation Approaches proposed by HyTrack}
\label{sec:hytrack-mitigations}
The HyTrack authors outline two potential mitigation strategies and discuss the trade-offs each entails with respect to their design goals.

\paragraph{Browser State Partitioning.}
Browser state partitioning would allow each app to use its own cookie storage and hence prevent cross-app tracking.
The seamless integration of web content remains intact, as no changes to the UI are necessary, but by completely removing the browser's shared state, benign uses like Single Sign-On (SSO) or ad personalization would be broken. 

\paragraph{Forced User Interaction.}
In contrast, Forced User Interaction avoids this problem by explicitly requiring user consent before launching a CT or TWA.
But this introduces a significant usability issue, as the user is forced to interact with the browser every time a web content is loaded, which not only degrades user experience but also breaks seamless integration of web content into the app.
Furthermore, this approach hands control and responsibility to the user, which is not ideal from a security perspective, as the user might be unaware of the consequences of their actions and may inadvertently enable tracking by failing to interact with the browser as required.

Other strategies, such as limiting CTs and TWAs to First-Party Domains or disabling them for specific domains via browser options ultimately reflect the aforementioned approaches, relying on either browser state partitioning or forced user interaction.
Therefore, these are not effective countermeasures against HyTrack.

\section{Capabilities and Capability-based Security}
\label{sec:capabilities}

% Capability defintion taken from my report
A capability is an unforgeable and tamperproof token of authority that grants its holder specific access rights to a protected object.
Unlike traditional access-control lists (ACLs), which base decisions on user or process identity, capabilities combine an object reference with an associated permission set, thereby enabling object-centric and decentralized access control.
A capability system enforces the principle of least privilege (PoLP) by ensuring that possession of a capability is both necessary and sufficient for performing an operation on the referenced object.
This follows the classic understanding established in early capability systems such as Hydra ~\cite{wulf1974hydra} and EROS ~\cite{shapiro1999eros}, where capabilities are described as ``prima facie evidence of authority''.

Because the right to access is embodied in the capability itself, no central authority needs to be consulted at the moment of use.
Classical capability systems even allow controlled delegation by transferring capabilities between processes.
In contrast, Byetrack deliberately prohibits capability transfer altogether, ensuring that capabilities remain bound to the originating application.
Different implementations vary in how capabilities are stored, propagated, and revoked, yet all enable finer-grained control over authority than identity-based approaches.


\section{Our Approach}
\label{sec:our-approach}

In the context of this thesis, capability-based control offers an elegant way to isolate shared browser state and restrict the propagation of cookies, ensuring that only entities explicitly possessing a valid capability may access a particular storage domain.
The weaknesses are addressed as follows:

\begin{itemize}
  \item \textbf{Explicit Cookie Isolation:} Cookies are stored only if a capability for the corresponding domain exists.  
    Based on the access rights granted by the capability, cookies are either stored in the shared jar or returned to the app for isolated local storage.
  \item \textbf{App-Aware Browser Context:} Each capability encodes the app's identity and version, enabling the browser to enforce per-app cookie policies and invalidate outdated tokens.
  \item \textbf{Capability-Scoped Access Control:} Third-party domains without valid capabilities cannot access shared state, thereby blocking cross-app tracking.  
    Legitimate use cases such as Single Sign-On (SSO) remain supported if the app developer explicitly allows them in the policy.
\end{itemize}

\section{Threat Model}
\label{sec:threat_model}
% Describe what components are trusted/untrusted / State assumption of HyTrack again (?) and our approach
The HyTrack attack consists of three main parties: the app developer, the tracking company providing a third-party library, and the user.
The tracking company aims to track the user across multiply apps and the web.
For this, it provides a library that the app developer can include in their app, advertising it as an analytics or advertising SDK.
The developer of an Android application unknowingly includes this library, which under the hood employs the HyTrack technique to conduct the tracking.

We want to prevent this library from tracking the app's user across multiple apps and empower app developer to use any third-party library without risking user privacy regarding cross-app tracking via HyTrack.

For this, we consider the app developer as benign but privacy-unaware.
The app itself is untrusted after installation, as it includes the malicious tracking library, which can include arbitrary code with the same privileges.
We assume the installer and the browser are trusted, as they initialize and enforce the mitigation.

As we hook our defense in the AndroidX browser library ~\cite{android:androidxbrowser}, any developer that wants to use the malicious tracking library or any other library that relies on CTs or TWAs automatically uses our framework if they configure and provide a policy file.
Therefore, only the AndroidX Browser library needs to be updated, rather than requiring developers to include an additional mitigation library --- a step that is frequently forgotten or deliberately omitted, as shown in prior work~\cite{derr2017keep,huang2019up}.

 % Background

%\chapter{The Threat Model}
% Describe what components are trusted/untrusted / State assumption of HyTrack again (?) and our approach

% - Attacker goals (cross-app tracking through shared cookies)
% - Attacker capabilities (can embed third-party libraries, open CT/TWA, use web storage)
% - Trusted components (browser, OS integrity, capability signing)
% - Assumptions (user installs apps intentionally; developer may misconfigure but not maliciously)
% - Out-of-scope threats (fingerprinting, malicious browsers, network-level adversaries)

%\section{The Threat Model}

The developer of an Android application unknowingly includes a third-party library that uses the HyTrack technique for their own purposes, such as advertising.
We want to prevent this library from tracking the apps user across multiple apps and empower the app developer to use any third-party library without risking user privacy in regards to cross-app tracking via HyTrack.

For this, we assume that the app developer is not malicious and does not intend to violate user privacy. 
Otherwise, developers could simply choose to omit using our mitigation framework and directly use the HyTrack library on their will.

A trusted component is the installer. Next to installing the app, it also extracts the app's policy and hands it of to the (trusted) browser, the Polcy Enforcement Point (PEP).
The browser initially generates the capability tokens according to the app's policy and sends them to the app, which stores them in private storage.

As the tracking library is included in the app, it has the same permissions as the app itsef, which means it can include arbitrary code, for example attempt to modify tokens or policies.
Additionally, we have to assume collaboration between the tracking library and other apps to share stored tokens and meta data of the mitigation framework.
Attemps such as sending policy to their own benefit and thus circumventing the mitigation are also possible.

As we hook our defense in the androidx browser library, any developer that wants to use the malicous tracking library -- or any other library that relies on Custom Tabs or Trusted Web Activities -- automatically uses our mitigatin framework. Thus, the developer cannot choose to omit the mitigation, but still disable it by not giving a policy at all.
Therefore, only the androidx browser library needs to be updated, instead of relying on the developer to additionally include the mitigation library, which could be forgotten or omitted intentionally. %TODO cite research here
 % Threat Model

\chapter{Methodology: Preventing Cross-App Tracking in CTs and TWAs via Capabilities}
\label{chap:method}

\todo[inline]{see draft}

% The method you want to use to address the problem.
% Explicitly state verifiable hypotheses.


% Current Idea: see JWT like Approach sketch

% ----------------------------Methodology--------------------------

Custom Tabs (CTs) and Trusted Web Activities (TWAs) enable a seamless integration between apps and the web by sharing browser state — notably session cookies.
While this improves user experience, it also introduces a significant privacy concern: third-party libraries can exploit this shared browser state to track users across apps that embed them.
HyTrack, for example, requires only a single shared third-party library used across multiple apps to persistently identify and track users, circumventing typical browser or OS sandboxing.

Our \unsure{or rather "my"} solution seeks to preserve the seamless user experience of CTs/TWAs while enforcing cookie isolation across apps to prevent unauthorized cross-app tracking - particularly from embedded third-party libraries.

\section{Proposed Solution: Capability-Based Cookie Isolation}
To address the issue, we propose a developer-defined policy mechanism paired with cryptographically enforced capabilities that dictate how cookies are managed and isolated per app.

% Developer Policy Declaration
 Upon app installation, the installer sends a policy crafted by the devoloper to the browser that defines:
 \begin{itemize}
   \item A list of trusted web servers (e.g., the developer’s domains or select third parties).
   \item A set of predefined cookie names expected to be used by those servers.
 \end{itemize}

This policy is used to create capabilities, which wrap cookie metadata (e.g., app ID, domain, name, and rights) into a secure structure.
These capabilities are encoded and signed by the browser (similar to JWT-tokens) \unsure{how to put this here} and then to the app.

Capabilities serve as authorization tokens for cookie access and are either:
\begin{itemize}
  \item \textbf{Wildcard Capabilities}, that allows the app to set any cookie name and value to be "filled" later or
  \item \textbf{Predefined Capabilities}, which have a fixed cookie name according to the policy and can only be stored in app-specific cookie jars.
\end{itemize}

% App-to-Browser Communication via Capabilities
When the app opens a URL via CT or TWA, it includes a list of capabilities along with the regular intent with the target URL.

The browser parses and verifies each capability by checking the following fields:
\begin{itemize}
  \item \textbf{Signature} to ensure the capability was issued by the browser and has not been tampered with.
  \item \textbf{App ID} to validate the origin, i.e. to ensure the app is authorized to use the capability.
  \item \textbf{Domain} to ensure the correct destination.
  \item \textbf{App Version Number}: to ensure the policy has not changed, i.e. to identify potential policy changes.
  \item \textbf{Rights} to restrict the access scope of the app, such as whether it can request the browser to read cookies. This is necessary to prevent libraries from using the browser to read capability values, which could otherwise be exploited for tracking.
  \item \textbf{Global Jar Flag} to determine whether to use the shared or app-specific cookie jar.
\end{itemize}

Cookies are included in the request only if the capability passes all checks.

The communication between Browser and Webserver remains unchanged, i.e. the browser sends Request and Set-Cookie headers as usual and the webserver responds with an (customized) Response and possibly new Cookies.

The browser them matches received cookies to capabilities:
\begin{itemize}
  \item If the cookie name is predefined, only the value is updated in the capability.
  \item If it's new and a wildcard capability exists, the browser creates a new capability by copying the wildcard capability and setting the name and value accordingly.
\end{itemize}

Additionally, if the global jar flag is set, the cookie is stored in the browser's cookie jar and the capability is sent back to the app. Otherwise, the capability is directly returned to the app and stored in its private app jar.

If validation fails at any point, the browser ensures a failsafe default by creating a fresh state before sending the request to the webserver, making sure that no cookies are sent along with it. 

\section{Benefits} 
Besides eliminating the possibility of cross-app tracking and the by hytrack postulated goals, I see several additional benefits this approach offers:
\begin{enumerate}[label=B\arabic*)]
  \item \textbf{Fine-Grained Control}: Developers can specify which cookies are shared and which are isolated, allowing for a more tailored approach to privacy.
  \item \textbf{No Browser State for Apps}: The browser does not need to remember app capabilities — the app holds and re-sends them with each intent.
  \item \textbf{No Ambient Authority}: By avoiding the need for the browser to store app states, we minimize the attack surface and potential vulnerabilities.
  \item \textbf{No Third-Party Code Changes}: The webserver does not need to be aware of the capabilities or make any changes to its code, as the browser handles the capability management.
  \item \textbf{Backwards Compatibility}: With minor adjustments, \dots \change{What has to be done exactly?}
\end{enumerate}

% -----------------------------------------------------------------

% Hypotheses
\section{Hypotheses}
My evaluation is guided by the following hypotheses:
\begin{enumerate}[label=H\arabic*)]
  \item Sending capabilities with each intent is sufficient to ensure that the browser can validate and manage cookies effectively.
  \item The installer can send the developer-defined policy to the browser and can return it to the app.
  \item The approach adheres to the goals of HytTrack (stated in \ref{chap:eval}).
  \item The approach fully eliminates cross-app tracking. \unsure{is this a Hypothesis?}
\end{enumerate}

% -----------------------------------------------------------------
 % Methodology

\chapter{Implementation}

% Document proof-of-concept in detail:
% 
% - Describe project components:
% 
%   - Modified AndroidX Browser (launchUrl hook).
% 
%   - Modified GeckoView / Fenix (policy enforcement, token verification).
% 
%   - Installer app (policy extraction and delivery).
% 
%   - Test apps (trusted vs untrusted use cases).
% 
% - Show code-level architecture or UML overview.
% 
% - Describe technical challenges (AIDL integration, JNI bridging, content providers, duplicate class conflicts, etc.) and how you solved them.
% 
% - Highlight design decisions: choice of token format, policy schema, storage mechanisms, ...

% -----------------------------Implementation---------------------------

To demonstrate the feasibility of our proposed mitigation strategy, we developed a proof-of-concept installer application, a new library that ships the changes and modified the androix browser library to inject our capability tokens for each call to launch a Custom Tab (CT) or Trusted Web Activity (TWA).

In this proo-of-concept, we chose the firefox mobile browser (Fenix) to act as the policy enforcement point, but it could have been any other browser the authors of HyTrack~\cite{USENIX:Wessels:2025} found out to be vulnerable to their attack, such as Chrome or Brave. %TODO: cite and check!

We modified the proof-of-concept apps provided by the authors of HyTrack and provide a test application for more insight into the framework's behavior.
For completeness, we also provide a "evil" acting app that demonstrates what the HyTrack library included in tan app could do to circumvent the mitigation.

\textit{Put a diagram here that shows the components and their interactions!!!}

\section{Custom Installer}

\section{Fenix Browser}

\section{Libraries}

  \subsection{Byetrack/Mitigation} % Write Byetrack here? Prob. title change needed "Byetrack: Capabilities as a Solution ..."

  \subsection{AndroidX Browser}

\section{HyTrack Demo Apps}

  \section{TrackerOne}

  \section{Launcher}

\section{Test App}

\section{Evil App}

 % Implementation

\chapter{Evaluation}
\label{chap:eval}

% The evaluation you want to use to assess how well you solution works.
% Provide details on verifying the stated hypotheses (experiments, subjects, measures, ...)


% -----------------------------Evaluation---------------------------

To assess the effectiveness of our proposed mitigation strategy, I adopt the three primary goals identified by the authors of HyTrack as essential for any viable defense:

\begin{enumerate}[label=\arabic*)]
  \item \textbf{Support all features of the web platform:} The solution must allow applications to display fully functional web content, including support for cookies, JavaScript, and modern APIs.
  \item \textbf{Preserve seamless integration:} The user experience must remain uninterrupted.
This includes avoiding obtrusive permission dialogs and maintaining smooth transitions between native and web content.
\item \textbf{Enable controlled access to shared browser state:} While isolation is required to prevent cross-app tracking, legitimate scenarios such as Single Sign-On (SSO) should continue to work within the context of a single application.
\end{enumerate}

These criteria reflect the fact that HyTrack exploits standard Android behavior—specifically, the shared browser state exposed through Custom Tabs and Trusted Web Activities—rather than relying on unauthorized access or system vulnerabilities.
Therefore, naive approaches like disabling shared cookies entirely would break common use cases and are not acceptable.

To validate these hypotheses, I will build on the open-sourced measurement tooling and proof of concept applications provided by the authors of HyTrack.
Specifically, I plan to:
\begin{itemize}
  \item Replicate the original HyTrack experiments under controlled conditions using two unrelated Android apps that embed the tracking library (similar to the HyTrack demo).
  \item Instrument network traffic (e.g., using mitmproxy or Frida) to observe the in capabilities wrapped cookies and their interactions with the browser.
  \item Apply the mitigation framework and compare observed behavior against the baseline.
\end{itemize} \unsure{Use itemize again?}

I will collect and analyze the following metrics:
\begin{itemize}
  \item Number of Capabilities created and used by the browser.
  \item The latency of framework operations, such as creating and validating Capabilities, in comparison with the baseline.
  \item \dots \change{add more}
\end{itemize}

In doing so, I aim to demonstrate that the solution effectively blocks HyTrack's cross-app tracking channel while maintaining compatibility and usability along with introducing the possibliity for the developer to send cookies in a fine-grained manner. \unsure{how to best write this?}

% Adhere to the goals of HyTrack
% Use proof of concept apps and open-sourced measurement tooling

% ------------------------------------------------------------------
 % Evaluation

\chapter{Discussion}

% Critically analyze results:
% 
% - How effectively does the approach meet the HyTrack goals?
% 
% - What are the trade-offs (complexity, compatibility, user transparency)?
%
% - How could Android or browsers adopt this natively?
%
% - Limitations (e.g., malicious developers, non-cookie tracking).
 % Discussion

\chapter{Related Work}
\label{chap:related_work}

% A discussion of the related work that has been conducted before.
% Discuss how your proposed work is related (and how it advances the state of the art).

% Notes:
% - Webtracking "Einordnung" in general!
% - Attempt of mitigation directly related

%Browser-level mitigations (CHIPS, ...)
%Android security frameworks (App sandboxing, DALs).

% --------------------Related Work--------------------------

HyTrack \cite{USENIX:Wessels:2025} demonstrates a novel cross-app and cross-web tracking technique in the Android ecosystem by exploiting the shared cookie storage between Custom Tabs (CTs) and Trusted Web Activities (TWAs).
This allows persistent tracking of users across multiple applications and the browser, even surviving user efforts to reset or sanitize their environments.

% --------------------Tabbed-Out-Custom-Tabs-as-evidence------------------------
The need to address HyTrack becomes even more critical in light of additional research on Custom Tabs.
Beer et al.~\cite{10646644} conducted a comprehensive security analysis of CTs and revealed that they can be exploited for state inference, SameSite cookie bypass, and UI-based phishing attacks.
Their work further shows that Custom Tabs are widely adopted, with over 83\% of top Android apps using them, often via embedded libraries.
These findings reinforce that CTs are a high-value attack surface and that the shared browser state --- central to HyTrack --- has broader security implications.
As TWAs are a specialized form of CTs, they are similarly affected, further enabling the tracking to be fully disguised.

While HyTrack highlights a serious privacy vulnerability, no concrete mitigation has been proposed that balances privacy with the legitimate need for seamless web integration within mobile apps, such as Single Sign-On or ad delivery.
This can be seen by taking a closer look at the two possible mitigation strategies discussed by the authors, namely Browser State Partitioning and Forced User Interaction.

Modern browsers prominently adopt state partitioning to combat third-party tracking.
Firefox's Total Cookie Protection (TCP)~\cite{mozillacookies} and Safari's Intelligent Tracking Prevention (ITP)~\cite{safariitp} both enforce per-site cookie jars, thereby limiting cookie-based cross-site tracking.
However, this also breaks legitimate third-party services that rely on shared cookies, such as SSO or ad personalization.

Google is actively working on a similar mechanism under the name CHIPS (Cookies Having Independent Partitioned State)~\cite{googlechips}.
CHIPS allows third-party cookies to be partitioned by the top-level site with an optional \texttt{Partitioned} flag, enabling legitimate services like SSO to maintain function while avoiding broad tracking vectors.
However, CHIPS is not applicable to Android's embedded web contents like CTs or TWAs, as the top-level site can be the tracker itself.
Our solution can be seen as extending this paradigm to the app level.

% --------------------------JWT-Tokens-Relation--------------------------------
Our interpretation of capability tokens is inspired by JSON Web Tokens (JWTs)~\cite{rfc7519}, which are widely used in web authentication to encode claims about a user or a session in a secure, verifiable manner.
Instead of storing user information directly on the server upon receiving a POST request, JWTs allow the server to issue a signed token that contains the necessary claims, which the client can then present in subsequent requests.
As a result, the server does not need to maintain session state, as the token itself carries all the information needed and can verify via the signature that the token has not been tampered with.
For this purpose, JWTs consist of three components separated by dots: a header that specifies the token type and algorithm for encoding and decoding it, a payload for the actual data, and a signature of the first two parts after base64 encoding that ensures the integrity of the token.
Our approach extends this idea by including the cookie information and other metadata in the token's payload, and by establishing a communication channel between the browser and the app: % -> cont. next line
the browser issues these tokens according to the app's policy, and the app presents them in subsequent requests to either access the browser's shared cookie jar or store cookies in its own app-local storage.

% --------------Security-of-Hybrid-Web/Mobile-Apps----------------------------
The work of Georgiev et al. titled "Breaking and Fixing Origin-Based Access Control in Hybrid Web Applications"~\cite{georgiev2014breaking} highlights critical failures in how hybrid apps enforce origin boundaries.
Specifically, they show that WebViews and hybrid frameworks often bypass or misapply the Same-Origin Policy (SOP), enabling attackers to inject or reuse authentication tokens across apps and domains.
Their proposed mitigation involves reintroducing stricter origin enforcement tied to app identities.
Our approach builds on this idea by using capability tokens to encode both the origin and the app context explicitly, thereby preventing unauthorized reuse or delegation.

 % Related Work

\chapter{Future Work}
\label{chap:future_work}

% ---------------------------Future Work----------------------------

    \info{Maybe there exist a trusted authority that can declare a domain as trusted e.g. Facebook for SSO etc. for default policy? predefined policies?} 

% ------------------------------------------------------------------
 % Future Work

\chapter{Conclusion}

% Summarize the achieved goals, contributions, and what the results imply for Android privacy.

 % Conclusion

%\chapter{Open Science}

To promote transparency and reproducibility in our research, we have made all relevant artifacts publicly available.
This includes the source code for our mitigation framework, the modified fenix browser and a test application demonstrating the functionality, next to the by HyTrack proof-of-concept applications instrumentalized with our mitigation.


SOMETHING ABOUT LICENSES?
 % Open Science

  % Additional "chapters" for Thesis Proposal
%\chapter{Schedule}

\todo[inline]{Write the schedule.}

% A schedule with verifiable milestones to be reached on a specified date.
% State potential risks for the proposal and their impact, as well as how to mitigate them, including alternatives

%\chapter{Success Criteria}

\todo[inline]{Write the success criteria.}

% Explicit sucess criteria that will help to assess your thesis, as a list of items:
% - Must-have criteria: Things your thesis must cover to be successful.
% - May-have criteria: Things your thesis can cover to improve its value.
% - Must-not-have criteria: Things your thesis will not cover (although they may thing so)

% -------------------------------Success Criteria----------------------------

The following list of success criteria will help assess the proposed solution's effictiveness and demonstrate the works' value \unsure{right word?}.

\begin{itemize}
  \item \textbf{Must-have criteria:} The solution must effectively prevent or limit cross-app tracking by blocking unauthorized cookie reuse across applications using the same tracking domain (i.e., blocks HyTrack-style tracking) while adhering to the outlined goals in x \change{reference to evaluation}.
  \item \textbf{May-have criteria:} \dots
  \item \textbf{Must-not-have criteria:} \dots
\end{itemize}

% ---------------------------------------------------------------------------


%% ----------------------------------------------------------------
\backmatter

%% ----------------------------------------------------------------
\label{Bibliography}
%\nocite{*}
%\lhead{\emph{Bibliography}}  % Change the left side page header to "Bibliography"
%\  % Use the "unsrtnat" BibTeX style for formatting the Bibliography
%\bibliographystyle{unsrtdin}
\bibliographystyle{IEEEtran}
\bibliography{src/Bibliography}  % The references (bibliography) information are stored in the file named "Bibliography.bib"

%% ----------------------------------------------------------------
% Now begin the Appendices, including them as separate files

\appendix % Cue to tell LaTeX that the following 'chapters' are Appendices

% Appendix A
\chapter{Appendix}

% \blindtext[2]

 % tables
\section{Example Policy File}
\label{appendix:policy}

\begin{verbatim}
{
  "predefined": {
    "global": {
      "royaleapi.com": ["__royaleapi_session_v2", "another_cookie"]
    },
    "private": {
      "schnellnochraviolimachen.de": ["named_cookie"],
      "royaleapi.com": ["__royaleapi_session_v2"]
    }
  },
  "wildcard": {
    "global": [
      "royaleapi.com"
    ],
    "private": [
      "nr-data.net"
    ]
  }
}
\end{verbatim}

\begin{itemize}
  \item \textit{royaleapi.com} only receives a predefined private and a global wildcard token (for general cookie usage).
    This is because the identical cookie \textit{\_\_royaleapi\_session\_v2} of the same domain is registered to receive a token for both isolation scopes.
    The token generator therefore downgrades the token to the private one, as it is more restrictive.

  \item \textit{schnellnochraviolimachen.de} receives a private predefined token limited to one cookie.

  \item \textit{nr-data.net} receives a private wildcard token, granting limited cookie handling rights without predefined cookie names.
    In this scenario, \textit{nr-data.net} is a third-party domain embedded in the first-party website \textit{royaleapi.com}.
\end{itemize}
 % example policy file
\section*{Use of Generative Digital Assistants}

% Concerning the use of generative digital assistants (e.g. ChatGPT or CoPilot) that go beyond spell/grammar checks the student is allowed to use such tools, but only for the following applications: 
% - code generation         (yes)
% - literature research     (yes)
% - text rewriting/revision (yes)
% - text generation         (no)
% If the student is allowed to use such generative tools, the student must explicitly indicate in an appendix which parts of the text and/or code were generated by which digital assistant.

% -------------------------------------------------------------------

used Claude Sonnet Model 4.0 embedded in Visual Studio Code exclusively for understanding the firefox codebase and to help linking my code additions together.

Models like ChatGpt and Claude also used for debugging purposes (copy paste and let it try to fix the code or to explain obscure error messages).

...
 % use of generative digital assistants

\listoftodos[Notes]
\end{document}  % The End
%% ----------------------------------------------------------------
