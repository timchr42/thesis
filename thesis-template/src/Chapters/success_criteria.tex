\chapter{Success Criteria}
\label{chap:success_criteria}

% Explicit sucess criteria that will help to assess your thesis, as a list of items:
% - Must-have criteria: Things your thesis must cover to be successful.
% - May-have criteria: Things your thesis can cover to improve its value.
% - Must-not-have criteria: Things your thesis will not cover (although they may thing so)

% -------------------------------Success Criteria----------------------------

The following criteria define the scope of this thesis and provide measurable goals to assess its success.

\section{Must-have criteria}

\begin{itemize}
  \item The solution must effectively prevent or significantly limit cross-app tracking via the shared browser state, while preserving the core functionality of Custom Tabs and Trusted Web Activities (e.g., Single Sign-On).
  \item It must adhere to the three primary goals outlined by HyTrack: (1) support all features of the Web platform , (2) do not break the seamless integration, and (3) make shared state available to the Custom Tab or Trusted Web Activity. \info{directly from HyTrack paper}
  \item A working proof of concept must be implemented to demonstrate the feasibility of the approach and to reproduce and compare against the HyTrack attack methodology.
  \item The thesis must provide a thorough discussion of how capability-based access control contributes to mitigating the identified tracking risks, including analysis of potential trade-offs (e.g., ambient authority). \info{Because concept of Capability technically not needed for implmentation but makes solution nicer and helps to circumvent Ambient Authority.}
\end{itemize}

\section{May-have criteria}

\begin{itemize}
  \item Provide an analysis of the solution’s impact in terms of performance, usability, integration effort for developers, and potential security limitations. \info{anything else?}
  \item Evaluate the framework across multiple Chromium-based browsers to assess generalizability.
  \item Investigate whether enhancements to the Digital Asset Links (DAL) mechanism could further strengthen the solution or propose a good alternative. \info{If not already necessary in solution.}
\end{itemize}

\section{Must-not-have criteria}
\begin{itemize}
  \item The thesis will not produce a production-ready implementationk, as the focus is on proof-of-concept and feasibility.
  \item The solution is not intended to prevent all forms of cross-app tracking, particularly not against malicious developers who intentionally bypass protections. The goal is to protect well-meaning developers from inadvertently integrating tracking libraries.
  \item It will not attempt to mitigate stateless tracking methods such as fingerprinting, which are outside the scope of shared state via cookies.
\end{itemize}

% ---------------------------------------------------------------------------
