\chapter{Future Work}
\label{chap:future_work}

% Policy evolution and distribution mechanisms.

% Integration with Digital Asset Links or Android permissions.

% Applicability to WebView, other browsers, or desktop environments.

% Maybe there exist a trusted authority that can declare a domain as trusted e.g. Facebook for SSO etc. for default policy? predefined policies?

% think about how to update the policy, expiration via nonce etc.

% revocation strategies?

% Developer toodling: how to create/test policies before deployment?

% Web-based policy editor?

% Capability Caching or capability handles/references instead of full tokens for performance optimization

% Generalizition beyond android

%  Eric Ackermann's Idea: 
      %HyTrack without cookies, but rather exploits the fact that the browser's cache is shalso shared across apps.
      %Combine this with hash function to create a unique identifier for the user + image in response with unique name.
      %If Webserver receives a request, it nows that the user has visited the page before and can use the image that encodes the uniue identifier (?).
      %Otherwise, the server can send a new image with a new unique identifier.

      %Solution:
      %Extens capability system to also isolate the browser cache. 
      %Trusted domains -> access to browser cache; Untrusted domains -> no access to browser cache.

      %=> Also interesting for optimitzaation of the capability system later on.

% Major Problem: Android's Custom Tabs don’t preserve or expose the caller identity reliably => No way of retrieving the caller identity in the browser

    % Tried the following approaches:

    % String callingPackage = null;

    % // Method 1: getReferrer() - most reliable for modern Android (API 22+)
    % if (Build.VERSION.SDK_INT >= Build.VERSION_CODES.LOLLIPOP_MR1) {
    %   Uri referrer = getReferrer();
    %   if (referrer != null) {
    %     if ("android-app".equals(referrer.getScheme())) {
    %       callingPackage = referrer.getHost(); // Package name from android-app://packagename
    %     } else {
    %       callingPackage = referrer.toString(); // Full referrer URI
    %     }
    %   }
    % }

    % // Method 2: getCallingPackage() - fallback (often null for external intents)
    % if (callingPackage == null) {
    %   callingPackage = getCallingPackage();
    % }

    % // Method 3: Intent referrer extra (fallback)
    % if (callingPackage == null && intent.hasExtra(Intent.EXTRA_REFERRER)) {
    %   Uri referrerExtra = intent.getParcelableExtra(Intent.EXTRA_REFERRER);
    %   if (referrerExtra != null && "android-app".equals(referrerExtra.getScheme())) {
    %     callingPackage = referrerExtra.getHost();
    %   }
    % }

    % // Method 4: Intent referrer name extra (another fallback)
    % if (callingPackage == null && intent.hasExtra(Intent.EXTRA_REFERRER_NAME)) {
    %   String referrerName = intent.getStringExtra(Intent.EXTRA_REFERRER_NAME);
    %   if (referrerName != null && referrerName.startsWith("android-app://")) {
    %     callingPackage = referrerName.substring("android-app://".length());
    %   }
    % }
% ---------------------------Future Work----------------------------





% ------------------------------------------------------------------
