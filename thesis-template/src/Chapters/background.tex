\chapter{Background}

% Provide the necessary technical and conceptual background:

% - Overview of Custom Tabs (CTs), Trusted Web Activities (TWAs), and browser cookie sharing on Android.

% (- Explain HyTrack in more detail (summarize its attack flow, key findings, and impact) — this links to your implementation context.)

% (- Introduce Capabilities/Tokens?)

\section{Capabilities}
%What are Capabilities?
%how to classify our capabilities -> identity-based cryptographic

Our system builds on the concept of capabilities -- unforgeable tokens that grant their holder specific rights to access a resource.
Unlike traditional access control lists (ACLs), capabilities enable a decentralized and fine-grained access model, as the right to perform an operation is encoded directly within the token rather than managed by a central authority.  
This means the browser can enforce cookie access based on the capabilities presented by the app, without maintaining a global permission mapping of app identities to cookies.
Byetrack applies this principle to browser state management: cookies are encapsulated within capability tokens that control which domains retain access to the shared cookie jar and which are isolated per app.
Therefore, we turn the browser into the Policy Enforcement Point (PEP), while the app, guided by a developer-provided policy, receives only the capabilities specified it present to the browser when opening a CT or TWA.


\section{Custom Tabs and Trusted Web Activities on Android}

To integrate web content into Android applications, developers can use several mechanisms that differ in terms of security, performance, and user experience.
Among these, Custom Tabs (CTs) and Trusted Web Activities (TWAs) have emerged as the most popular alternatives to traditional WebViews, offering better performance and tighter integration with the user's default browser.

Custom Tabs were introduced to allow apps to display web content within the app's interface while leveraging the full capabilities of the user's browser.
Unlike a WebView, which runs a separate, minimal web engine within the app, a Custom Tab is rendered by the installed browser itself.
This means that all browser features -- such as optimized rendering, password managers, saved credentials, and cookies -- remain available.
Developers can also customize the browser's UI elements, such as toolbar color and menu items, to visually align the Custom Tab with their app's theme.
As a result, users perceive a seamless transition between native and web content without leaving the app context.

Trusted Web Activities (TWAs) extend this concept further by removing nearly all browser UI elements, including the URL bar, and displaying web content in full-screen mode.
This allows developers to integrate entire Progressive Web Apps (PWAs) or other web-based experiences into their native apps while maintaining a consistent appearance.
For security reasons, launching a TWA requires a Digital Asset Link (DAL) -- a mutual verification between the app and the website -- ensuring that both belong to the same trusted party.
If this trust relationship cannot be verified, Android automatically downgrades the TWA to a regular Custom Tab.

A key advantage of both CTs and TWAs is that they share the browser's state.
This means users can stay logged in to websites, reuse stored cookies, and maintain personalized settings across different apps and browsing sessions.
This behavior improves usability and supports features like Single Sign-On (SSO), as authentication tokens from the browser can be reused within an app's embedded web view.
However, as later discussed in Section \ref{sec:hytrack}, this same feature also introduces significant privacy risks.
The shared cookie storage allows any app -- intentionally or not -- to access browser state information used by others, thereby enabling persistent cross-app and cross-web tracking techniques such as HyTrack.

In summary, Custom Tabs and Trusted Web Activities offer a powerful bridge between the native and web ecosystems on Android.
They combine the convenience and functionality of a full browser with the visual coherence of an app-embedded experience.
Yet, the same integration that improves usability also blurs traditional security and privacy boundaries between apps and the web, which is exploited by for persistent-cross app tracking in the form of the HyTrack attack.

\section{HyTrack Attack Overview}
\label{sec:hytrack}
HyTrack, introduced by Wessels et al. \cite{USENIX:Wessels:2025}, exposes a fundamental privacy flaw in this shared-state model.
It demonstrates that third-party libraries embedded in multiple apps can exploit the browser's global cookie storage to identify and track users across applications and even into their normal web browsing sessions.
By leveraging standard Android features—rather than any explicit vulnerability—HyTrack highlights how the very mechanisms designed to make CTs and TWAs seamless for users can also undermine Android's app isolation guarantees.

Whenever an app opens a CT or TWA to display web content, the request is executed within the context of the user’s default browser.
This means that all cookies set by the visited domain are stored in the browser's global cookie jar and automatically reused in subsequent sessions — even if they originate from different apps.
While this shared state enables seamless Single Sign-On and personalization, it also allows a tracking entity to correlate activity across multiple apps that interact with the same web domain.

HyTrack leverages this design as follows: a seemingly benign third-party library, included in several independent apps, silently opens a CT or TWA to a tracking domain controlled by the library’s author.
When this web page is first loaded, the server sets a unique identifier in a cookie, which is then stored in the shared browser state.
When another app using the same library later opens a CT or TWA to the same tracking domain, the browser automatically attaches the existing cookie, thereby revealing that both apps are used by the same user.
This creates a powerful cross-app identity link that persists outside Android’s app sandbox and is invisible to both users and app developers.

Even more concerning, HyTrack's tracking identifiers are resilient to deletion.
Because Android’s automatic backup mechanisms restore application data, including browser-managed cookies, the tracking identifier can survive browser resets, app reinstallations, and even factory resets.
In effect, HyTrack achieves evercookie-like persistence at the system level, reviving deleted identifiers upon device restoration.

The feasibility of this attack stems from three fundamental weaknesses in the current CT and TWA model:

\subsection{Weaknesses in Custom Tabs and Trusted Web Activities}
\label{sec:hytrack-weaknesses}
\begin{enumerate}[label=W\arabic*)]
  \item 
    \textbf{Implicit and Persistent Cookie Sharing:}
    All apps using CTs or TWAs share a single, persistent global browser cookie jar, regardless of developer intent.
    This shared state persists across app launches and user attempts to clear tracking data.

  \item
    \textbf{Lack of App Context in the Browser:}
    The browser has no knowledge of which app initiated a given request and therefore cannot enforce app-specific cookie isolation or policy controls.

  \item
    \textbf{Unrestricted Third-Party Inclusion:}
    Any third-party library embedded across multiple apps can open CTs or TWAs, gaining access to the shared browser state and enabling tracking across unrelated apps.
\end{enumerate}

By exploiting these design characteristics, HyTrack bridges the isolation between native and web contexts, effectively transforming legitimate web-integration features into a cross-app tracking channel.

\subsection{Design Goals and Motivation}
\label{sec:hytrack-goals}

The authors of HyTrack identified three essential goals that any practical mitigation against cross-app tracking must fulfill~\cite{USENIX:Wessels:2025}.
We adopt these goals as guiding principles for the design of our capability-based framework:

\begin{enumerate}[label=G\arabic*)]
  \item \textbf{Support for Web Platform Features:}
  Any mitigation should preserve the full functionality of web content, including support for cookies, JavaScript, and modern browser APIs.

  \item \textbf{Seamless Integration:}
  The mitigation must operate transparently, without requiring additional user permissions or altering the normal app and browser experience.

  \item \textbf{Controlled Access to Shared Browser State:}
  Isolation between applications must prevent tracking via shared state, while still allowing legitimate sharing scenarios such as Single Sign-On (SSO).
\end{enumerate}

\subsection{Mitigation Approaches}
\label{sec:hytrack-mitigations}
The HyTrack authors discuss two potential mitigation strategies and highlight their resspective trade-offs in respect to their design goals.

\paragraph{Browser State Partitioning.}
Browser state partitioning would allow each app to use its own cookie storage and hence prevent cross-app tracking.
The seamless integration of web content remains intact, as no changes to the UI are necessary, but by completely removing the browser's shared state, benign uses like Single Sign-On (SSO) or ad personalization would be broken. 

\paragraph{Forces User Interaction.}
In contrast, Forced User Interaction avoids these problems by allowing the browser to use its shared cookie storage.
But this introduces a significant usability issue, as the user is forced to interact with the browser every time a web content is loaded, which not only degrades user experience but also breaks seamless integration of web content into the app.
Furthermore, this approach hands control and responsibility to the user, which is not ideal from a security perspective, as the user might be unaware of the consequences of their actions and may inadvertently enable tracking by failing to interact with the browser as required.

\paragraph{Other Strategies.}
Other strategies, such as limiting CTs and TWAs to First-Party Domains or disabling them for specific domains via browser options ultimately reflect the aforementioned approaches, relying on either browser state partitioning or forced user interaction.
Therefore, these are not effective countermeasures against HyTrack.

\chapter{The Threat Model}
% Describe what components are trusted/untrusted / State assumption of HyTrack again (?) and our approach

% - Attacker goals (cross-app tracking through shared cookies)
% - Attacker capabilities (can embed third-party libraries, open CT/TWA, use web storage)
% - Trusted components (browser, OS integrity, capability signing)
% - Assumptions (user installs apps intentionally; developer may misconfigure but not maliciously)
% - Out-of-scope threats (fingerprinting, malicious browsers, network-level adversaries)

%\section{The Threat Model}

The developer of an Android application unknowingly includes a third-party library that uses the HyTrack technique for their own purposes, such as advertising.
We want to prevent this library from tracking the apps user across multiple apps and empower the app developer to use any third-party library without risking user privacy in regards to cross-app tracking via HyTrack.

For this, we assume that the app developer is not malicious and does not intend to violate user privacy. 
Otherwise, developers could simply choose to omit using our mitigation framework and directly use the HyTrack library on their will.

A trusted component is the installer. Next to installing the app, it also extracts the app's policy and hands it of to the (trusted) browser, the Polcy Enforcement Point (PEP).
The browser initially generates the capability tokens according to the app's policy and sends them to the app, which stores them in private storage.

As the tracking library is included in the app, it has the same permissions as the app itsef, which means it can include arbitrary code, for example attempt to modify tokens or policies.
Additionally, we have to assume collaboration between the tracking library and other apps to share stored tokens and meta data of the mitigation framework.
Attemps such as sending policy to their own benefit and thus circumventing the mitigation are also possible.

As we hook our defense in the androidx browser library, any developer that wants to use the malicous tracking library -- or any other library that relies on Custom Tabs or Trusted Web Activities -- automatically uses our mitigatin framework. Thus, the developer cannot choose to omit the mitigation, but still disable it by not giving a policy at all.
Therefore, only the androidx browser library needs to be updated, instead of relying on the developer to additionally include the mitigation library, which could be forgotten or omitted intentionally. %TODO cite research here

% -------------------------------------------------Draft Capability Section -------------------------------------

%\section{Capabilities}
%We try to leverage the nature of capabilities as unforgeable tokens that grant the holder specific access rights to a resource.

