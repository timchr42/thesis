\chapter{Related Work}
\label{chap:related_work}

% A discussion of the related work that has been conducted before.
% Discuss how your proposed work is related (and how it advances the state of the art).

% Notes:
% - Webtracking "Einordnung" in general!
% - Attempt of mitigation directly related

% --------------------Related Work--------------------------

\todo[inline]{
  - Explain "related work" hytrack again \\
  - Also other works like JWT etc. contributing to method \\
}
Tracking mechanisms are typically divided into two broad categories: stateful and stateless tracking.
Stateless tracking, also known as fingerprinting, infers a user's identity based on a combination of device-specific attributes.
Consequently, this method is hard to detect and block, but is also inherently less reliable, as small system changes may alter the fingerprint and disrupt identification

Instead, Stateful tracking relies on storing unique identifiers on the client device, most commonly through cookies or local storage.
When a user revisits a site or interacts with embedded third-party content across domains, these identifiers are sent along with requests, allowing persistent recognition.
While straightforward and highly effective, stateful tracking has become increasingly restricted through browser policies (e.g., third-party cookie blocking) and mobile platform changes such as the ability to disable the Google Advertising ID (GAID) on Android.

Despite mitigation efforts, stateful tracking techniques are now re-emerging in new contexts.
A notable example is HyTrack \cite{USENIX:Wessels:2025}, which demonstrates a novel cross-app and cross-web tracking technique in the Android ecosystem.
HyTrack exploits the shared cookie storage between Custom Tabs and Trusted Web Activities (TWAs) to persistently track users across multiple applications and the browser, even surviving user efforts to reset or sanitize their environments.
While HyTrack highlights a serious privacy vulnerability, no concrete mitigation has been proposed that balances privacy with the legitimate need for seamless web integration—such as Single Sign-On or ad delivery—within mobile apps.

% How it advances the state of the art
Our work addresses this gap by proposing a capability-based access control framework for Android applications using CTs and TWAs.
By wrapping Cookies into fine-grained capability tokens created by the browser according to the app developers policy, the browser decides which cookies are allowed to be stored in the by default shared cookie storage and which are stored in the app-specific storage and thus not accessible to third-party libraries.
As the shared cookie storage still exists for domains declared as first-party or trusted by the app developer, core functionality like seamless integration of web content \unsure{correct writing it like this} is preserved.
In contrast to prior work that focuses on browser-side or user-driven defenses (e.g., partitioned storage \info{like Safari} or consent prompts \info{current state when opening TWA}), our approach provides developers with a practical and enforceable way to render cross-app tracking infeasible while adhering to the goals stated by the HyTrack authors.

\todo[inline]{
  - no change of webservers code
  - elaborate on "Discussion" and mitigation idead stated in HyTrack \\
}


% ----------------------------------------------------------
