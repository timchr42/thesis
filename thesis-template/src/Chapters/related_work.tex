\chapter{Related Work}
\label{chap:related_work}

% A discussion of the related work that has been conducted before.
% Discuss how your proposed work is related (and how it advances the state of the art).

% Notes:
% - Webtracking "Einordnung" in general!
% - Attempt of mitigation directly related

%Browser-level mitigations (CHIPS, ...)
%Android security frameworks (App sandboxing, DALs).

% --------------------Related Work--------------------------

HyTrack \cite{USENIX:Wessels:2025} demonstrates a novel cross-app and cross-web tracking technique in the Android ecosystem by exploiting the shared cookie storage between Custom Tabs (CTs) and Trusted Web Activities (TWAs).
This allows persistent tracking of users across multiple applications and the browser, even surviving user efforts to reset or sanitize their environments.

% --------------------Tabbed-Out-Custom-Tabs-as-evidence------------------------
The need to address HyTrack becomes even more critical in light of additional research on Custom Tabs.
Beer et al.~\cite{10646644} conducted a comprehensive security analysis of CTs and revealed that they can be exploited for state inference, SameSite cookie bypass, and UI-based phishing attacks.
Their work further shows that Custom Tabs are widely adopted, with over 83\% of top Android apps using them, often via embedded libraries.
These findings reinforce that CTs are a high-value attack surface and that the shared browser state --- central to HyTrack --- has broader security implications.
As TWAs are a specialized form of CTs, they are similarly affected, further enabling the tracking to be fully disguised.

While HyTrack highlights a serious privacy vulnerability, no concrete mitigation has been proposed that balances privacy with the legitimate need for seamless web integration within mobile apps, such as Single Sign-On or ad delivery.
This can be seen by taking a closer look at the two possible mitigation strategies discussed by the authors, namely Browser State Partitioning and Forced User Interaction.

Modern browsers prominently adopt state partitioning to combat third-party tracking.
Firefox's Total Cookie Protection (TCP)~\cite{mozillacookies} and Safari's Intelligent Tracking Prevention (ITP)~\cite{safariitp} both enforce per-site cookie jars, thereby limiting cookie-based cross-site tracking.
However, this also breaks legitimate third-party services that rely on shared cookies, such as SSO or ad personalization.

Google is actively working on a similar mechanism under the name CHIPS (Cookies Having Independent Partitioned State)~\cite{googlechips}.
CHIPS allows third-party cookies to be partitioned by the top-level site with an optional \texttt{Partitioned} flag, enabling legitimate services like SSO to maintain function while avoiding broad tracking vectors.
However, CHIPS is not applicable to Android's embedded web contents like CTs or TWAs, as the top-level site can be the tracker itself.
Our solution can be seen as extending this paradigm to the app level.

% --------------------------JWT-Tokens-Relation--------------------------------
Our interpretation of capability tokens is inspired by JSON Web Tokens (JWTs)~\cite{rfc7519}, which are widely used in web authentication to encode claims about a user or a session in a secure, verifiable manner.
Instead of storing user information directly on the server upon receiving a POST request, JWTs allow the server to issue a signed token that contains the necessary claims, which the client can then present in subsequent requests.
As a result, the server does not need to maintain session state, as the token itself carries all the information needed and can verify via the signature that the token has not been tampered with.
For this purpose, JWTs consist of three components separated by dots: a header that specifies the token type and algorithm for encoding and decoding it, a payload for the actual data, and a signature of the first two parts after base64 encoding that ensures the integrity of the token.
Our approach extends this idea by including the cookie information and other metadata in the token's payload, and by establishing a communication channel between the browser and the app: % -> cont. next line
the browser issues these tokens according to the app's policy, and the app presents them in subsequent requests to either access the browser's shared cookie jar or store cookies in its own app-local storage.

% --------------Security-of-Hybrid-Web/Mobile-Apps----------------------------
The work of Georgiev et al. titled "Breaking and Fixing Origin-Based Access Control in Hybrid Web Applications"~\cite{georgiev2014breaking} highlights critical failures in how hybrid apps enforce origin boundaries.
Specifically, they show that WebViews and hybrid frameworks often bypass or misapply the Same-Origin Policy (SOP), enabling attackers to inject or reuse authentication tokens across apps and domains.
Their proposed mitigation involves reintroducing stricter origin enforcement tied to app identities.
Our approach builds on this idea by using capability tokens to encode both the origin and the app context explicitly, thereby preventing unauthorized reuse or delegation.

