\chapter{Methodology}
\label{chap:methodology}

\todo[inline]{Write the methodology after sync-up with Noah}

% The method you want to use to address the problem.
% Explicitly state verifiable hypotheses.


% Current Idea: see JWT like Approach sketch

% ----------------------------Methodology--------------------------

- Browser State Isolation with Shared Cookie Storage ONLY for trusted URLs \\
  -> still allows SSO for facebook but blocks Cookie reuse for tracking for ad.com \\
  -> ad.com can still use cookies (but different cookie jar per app) if insist needing cookie for functionality (better ads etc.) \\
  -> idea of Domain signed with different Domain Authentication Keys (Ambient Domain -> shared -> signed with same key)\\
- How to decide which URLs are trusted or not? \\
- Does Developer need to specify trusted URLs? Default Value -> blocked? and has to be requested by library? \\

\section{Hypotheses}

My evaluation is guided by the following hypotheses:

\begin{itemize}
  \item \textbf{H1}: The framework prevents a hytrack deployed third-party library from tracking users across different apps.
  \item \textbf{H2}: It does not break legitimate web interactions within a single app, including login flows and session persistence.
  \item \textbf{H3}: It preserves the seamless user experience expected from CTs and TWAs.
    \\ \vdots
  \item \textbf{Hx}: \dots
\end{itemize}

% -----------------------------------------------------------------
