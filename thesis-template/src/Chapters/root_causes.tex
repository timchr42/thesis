\label{chap:root_causes}
%\chapter{Root Causes of Cross-App Tracking in HyTrack and Mitigation} % or: Strategy
%
%HyTrack demonstrates that third-party libraries embedded in multiple apps can exploit shared browser state -- particularly cookies -- to identify and track users across different apps.
%This attack is possible due to implicit assumptions in current browser-embedding APIs, which fail to account for cross-app boundaries or developer intent.

\section{Root Causes of HyTrack and Our Mitigation Strategy}

The feasibility of fhyTrack relies on the following underlying weaknesses in the CT and TWA security model:

\begin{itemize}
  \item \textbf{Implicit and Persistent Cookie Sharing}: All apps using CTs or TWAs inadvertently access a single, persistent global browser cookie jar, regardless of whether such sharing is intended by the developer.
    This cookie state persists across app launches and can survive typical user efforts to clear tracking data.
  \item \textbf{Lack of App Context in Browser}: The browser is unaware of which app initiated a given request, and therefore cannot enforce app-specific isolation or developer-defined policies.
    As a result, all apps using CTs or TWAs share the same cookie state, even if their purposes and trust levels differ.
  \item \textbf{Unrestricted Third-Party Inclusion}: Third-party libraries embedded in multiple apps gain access to the shared browser cookie jar, allowing them to identify users across app boundaries.
\end{itemize}

Our approach addresses these issues by introducing explicit, developer-defined policies and browser-enforced \textbf{capability tokens}:

\begin{itemize}
  \item \textbf{Explicit Cookie Isolation}: Cookies are stored only if the domain is explicitly declared as trusted or untrusted in the app's policy, i.e. a capability for the domain exists. 
    Based on the token(s) received by the app, the browser either stores the wrapped cookie in the shared cookie jar or returns it to the app for isolated local storage.
  This reverses the implicit sharing model and ensures that even persistent HyTrack identifiers cannot be used for cross-app tracking, as each app maintains a distinct, app-scoped identity -- assuming the policy is correctly configured.
  \item \textbf{App-Aware Browser Context}: Each capability encodes an \textit{App ID}, allowing the browser to enforce per-app cookie policies and prevent unintended delegation between apps.
  Additionally, capabilities include an \textit{App Version Number}, enabling the browser to detect app updates and invalidate tokens that no longer reflect the current policy.
  \item \textbf{Capability-Scoped Access Control}: Malicious third-party domains are confined to app-local storage and cannot access the shared cookie jar, thereby blocking cross-app tracking via embedded libraries. 
  At the same time, legitimate use cases, such as Single Sign-On (SSO), remain supported by granting trusted domains the necessary capabilities to access the shared cookie jar.
\end{itemize}

By making browser state access explicit, app-aware and scoped, our solution enforces cookie isolation across apps -- particularly against embedded third-party libraries -- thereby neutralizing HyTrack's core tracking mechanisms while preserving the seamless user experience of CTs and TWAs and supporting legitimate app-web integrations.
