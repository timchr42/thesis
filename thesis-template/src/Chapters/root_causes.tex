\label{chap:root_causes}
%\chapter{Root Causes of Cross-App Tracking in HyTrack and Mitigation} % or: Strategy
%
%HyTrack demonstrates that third-party libraries embedded in multiple apps can exploit shared browser state -- particularly cookies -- to identify and track users across different apps.
%This attack is possible due to implicit assumptions in current browser-embedding APIs, which fail to account for cross-app boundaries or developer intent.

\section{Root Causes of Cross-App Tracking and Our Mitigation Strategy}

The authors of HyTrack identified the following root causes that make such tracking possible:

\begin{itemize}
  \item \textbf{Implicit Cookie Sharing}: All apps using CTs or TWAs access a single, global browser cookie jar, regardless of whether such sharing is intended by the developer.
  \item \textbf{Lack of App Context in Browser}: The browser is unaware of which app initiated a given request, and therefore cannot enforce app-specific isolation or developer-defined policies.
  \item \textbf{Unrestricted Third-Party Inclusion}: Third-party libraries included in multiple apps gain access to shared cookies and can use this state to persistently identify users.
\end{itemize}

Our approach addresses these issues by introducing explicit, developer-defined policies and browser-enforced \textbf{capability tokens}:

\begin{itemize}
  \item \textbf{Explicit Cookie Isolation}: Cookies are stored only if the domain is explicitly declared as trusted or untrusted in the app's policy, i.e. a capability for the domain exists. 
    Depending on the token, the browser either stores the wrapped cookie in the browser's cookie jar or returns it to the app for local storage.
  This reverses the implicit sharing assumption.
  \item \textbf{App-Aware Browser Context}: Each capability includes an \textit{App ID}, allowing the browser to enforce per-app cookie policies and prevent unintended delegation across apps.
  Furthermore, the browser can detect app updates via an \textit{App Version Number} embedded in the capability, allowing it to reject outdated or invalid capabilities that no longer match the current policy.
  \item \textbf{Capability-Scoped Access Control}: Malicious third-party domains are restricted to app-local storage and cannot access the shared cookie jar, thereby preventing cross-app tracking via embedded libraries. 
  Legitimate use cases, such as Single Sign-On (SSO), remain supported by granting trusted domains the necessary capabilities to access the shared cookie jar.
\end{itemize}

By making browser state access explicit, scoped, and app-aware, our solution neutralizes HyTrack's core tracking mechanisms while maintaining compatibility with legitimate app-web integrations.
