\section{Use of Generative Digital Assistants}

% Concerning the use of generative digital assistants (e.g. ChatGPT or CoPilot) that go beyond spell/grammar checks the student is allowed to use such tools, but only for the following applications: 
% - code generation         (yes)
% - literature research     (yes)
% - text rewriting/revision (yes)
% - text generation         (no)
% If the student is allowed to use such generative tools, the student must explicitly indicate in an appendix which parts of the text and/or code were generated by which digital assistant.

% -------------------------------------------------------------------

%used Claude Sonnet Model 4.0 embedded in Visual Studio Code exclusively for understanding the firefox codebase and to help linking my code additions together.
%also helpful to pinpoint relevant files and functions when searching manually was too tedious (especially in the threading process)
%
%Models like ChatGpt and Claude also used for debugging purposes (copy paste and let it try to fix the code or to explain obscure error messages).
%
%Used GitHub Copilot auto-completion extensively while writing code to speed up the process and get suggestions for common patterns.
%
%Used ChatGpt for rewriting and improving my technical writing, to improve clarity, grammar and structure of sentences and paragraphs.
%
%Without these tools the project would have taken significantly longer to complete.

During this thesis, I used several generative AI tools within the allowed scope (code generation, literature lookup, debugging help, and text revision).

Claude Sonnet 4.0, integrated into Visual Studio Code, supported me in navigating and understanding the Firefox/GeckoView codebase.
It helped me locate relevant files, follow control flows, and identify functions involved in specific behaviors—tasks that would have been very time-consuming manually.
Claude was also useful when writing the JSON token parser and serializer in Java and C++, and when selecting and using appropriate cryptographic libraries (e.g., HMAC-SHA256, AES-CBC with random IVs).

Both ChatGPT (GPT-4/5 models) and Claude were used for debugging support by explaining unclear compiler or runtime errors and suggesting fixes. While some suggestions (especially related to JNI) turned out to be unhelpful, others were surprisingly effective.
Notably, ChatGPT helped point me toward ActivityStarter as the correct place in Android's source code to propagate the caller UID into an Intent.

GitHub Copilot was used for code completion and to speed up writing boilerplate code, common Android patterns, and repetitive structures.

For writing the thesis itself, I used ChatGPT to rephrase and improve technical sections for clarity, structure, and readability.

Overall, these tools helped speed up the development and writing process by reducing the time needed to search through large codebases, debug complex issues, and refine written text.
