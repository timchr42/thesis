\section{Use of Generative Digital Assistants}

% Concerning the use of generative digital assistants (e.g. ChatGPT or CoPilot) that go beyond spell/grammar checks the student is allowed to use such tools, but only for the following applications: 
% - code generation         (yes)
% - literature research     (yes)
% - text rewriting/revision (yes)
% - text generation         (no)
% If the student is allowed to use such generative tools, the student must explicitly indicate in an appendix which parts of the text and/or code were generated by which digital assistant.

% -------------------------------------------------------------------

%used Claude Sonnet Model 4.0 embedded in Visual Studio Code exclusively for understanding the firefox codebase and to help linking my code additions together.
%also helpful to pinpoint relevant files and functions when searching manually was too tedious (especially in the threading process)
%
%Models like ChatGpt and Claude also used for debugging purposes (copy paste and let it try to fix the code or to explain obscure error messages).
%
%Used GitHub Copilot auto-completion extensively while writing code to speed up the process and get suggestions for common patterns.
%
%Used ChatGpt for rewriting and improving my technical writing, to improve clarity, grammar and structure of sentences and paragraphs.
%
%Without these tools the project would have taken significantly longer to complete.

In the course of this thesis, I made use of several generative digital assistants strictly within the permitted scope (code generation, literature research, debugging support, and text rewriting/revision).

I used Claude Sonnet 4.0, embedded in Visual Studio Code, mainly to support my understanding of the Firefox/GeckoView codebase and to help integrate my own modifications into it.
This included locating relevant files, understanding complex control flows, and identifying functions involved in multi-threaded operations.
Claude was also used to clarify error messages and assist with debugging when manual investigation would have been overly time-consuming.

ChatGPT (GPT-4/5 models) and Claude were additionally used for debugging support by explaining obscure compiler or runtime errors and suggesting possible fixes.
At no point were these tools used to produce original scientific text; instead, they were used to rewrite, refine, and improve my own drafts to ensure clarity, conciseness, and correctness in technical sections.

During development, I used GitHub Copilot for code auto-completion, especially for boilerplate patterns, common Android structures, and repetitive code segments.
%Copilot served as an assistive tool to speed up development without generating novel architectural components.

Overall, these tools significantly accelerated the development and writing process by reducing the time spent searching through large codebases, debugging intricate issues, and polishing written text.
All conceptual contributions, implementation decisions, and written content originate from my own work.
